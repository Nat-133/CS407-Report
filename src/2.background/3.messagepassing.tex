\section{Message Passing Interface}
Currently, the paradigm for communication between nodes in distributed systems is message passing. Under this model, a set of processes have only local memory but are able to communicate with others through the sending and receiving of messages. Importantly, data transfer between process memories requires the explicit involvement of both processes. Some changes have been introduced in newer versions of MPI that allow for one sided messages but these are still restricted. They can only be received in certain “epochs” - periods of time on the remote processor that message must adhere to.